% -*- root: ../../main.tex -*- %

In the past years, enterprises reacted to their need for increased computational power by making use of \ac{IT} infrastructure and software applications that are offered as services. These offers are known as \ac{PaaS} and \ac{SaaS} \cite[p.~606]{Buyya2009Cloud}. To be able run software in an environment that features such services, various approaches have been presented \cite[p.~81]{Bernstein2014Containers}. One of them is the use of software containers.
Software containers provide a way of packaging and executing processes that isolates the application from the underlying \ac{OS} of a computer and other processes that run on it.

The concept of software containers is no new notion: an early predecessor was the \texttt{chroot} command, which dates back to 1979, \emph{software jails} followed in 1998 \cite[p.~82]{Bernstein2014Containers}.
In the second decade of the \nth{21} century, solutions like Rocket, LXD, and Docker emerged, which aim at the introduction of standardized, re-usable software containers, usually in combination with tools for their management. Among these solutions, Docker is very popular. In the beginning of 2016, Docker's repository was the \nth{20} most ``starred'' repository on the source code management platform \emph{GitHub} -- ranking four positions behind the Linux kernel repository \cite{Github2016Repositories}. Docker comes with a set of utilities, which extend its main container-related functionality.

Organizations perform temporal and logical sequences of actions that help to interact with business relevant entities -- business processes -- with the objective to reach their business goals. If these processes are coordinated in an automated way, they are also called \emph{workflows}. \acp{WfMS} are designed to support the definition, execution and monitoring of these workflows \cite{Becker1999Identifying,Hollingsworth1995Wfmc}.
With regards to the challenges that heterogeneous and distributed \ac{IT} environments as described above impose on \acp{WfMS} -- \eg the distribution of workflows to their location of enactment, the requirement to be able to adapt to increasing workload or manage the remote execution of tasks -- it could be of interest to fathom possible benefits that may arise from the use of the Docker tool set in the context of these \acp{WfMS}. The primary objectives of the thesis at hand are thus to address the following questions and to derive artifacts from the findings that may serve as a foundation for the conceptualization and implementation of future Docker-based \acp{WfMS}:

\begin{description}[nosep]
  \item[RQ1:] How can Docker leverage the deployment and execution of workflows in a distributed environment?
  \item[RQ2:] Which decisions in software architecture and software design of a WFMS are complemented by Docker's functionality?
\end{description}

Queries on the research portal Scopus with the search terms `docker AND wfms' and `docker AND ``workflow management system''' on November \nth{6} 2015 yielded no results for existing research. A search conducted during the implementation with the same queries on Google Scholar returned a publication by Zheng and Thain which is focused on using Docker to provide controllabe runtime environments for the execution of tasks in the Makeflow \ac{WfMS} \cite{Zheng2015Integrating}. Compared to its topic, the scope of the thesis at hand broader as it aims to find general uses for Docker in \ac{WfMS}.

The structure of this thesis follows the design science research process suggested by Peffers et al. \cite[pp.~89-92]{Peffers2007Design}. The research problem is identified in this very chapter and chapters \ref{cha:workflow_management_systems}, \ref{cha:docker} and \ref{cha:architecture_styles}, in which the fundamental concepts of \acp{WfMS} and Docker as well as architecture styles are introduced.
Based on considerations drawn from these concepts, the objectives of a solution are inferred and a prototype is designed in Chapter~\ref{cha:solution_design}. The design and implementation of the prototype is described in chapters \ref{cha:solution_design} and \ref{cha:implementation} respectively. In Chapter~\ref{cha:evaluation}, the developed mechanisms and the prototype are evaluated. Finally, the findings of this thesis are summarized and suggestions for subsequent research are presented in Chapter\ref{cha:conclusion}.
