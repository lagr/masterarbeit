% -*- root: ../../main.tex -*- %

\section{Concepts} % (fold)
\label{sec:concepts}

  \subsection{Workflow} % (fold)
  \label{sub:workflow}
    In order to achieve their business goals, organizations perform temporal and logical sequences of tasks that help to interact with business relevant entities. These sequences are known as \emph{business processes}. If the logic that controls the processes is performed in an automated way, \eg by an information system, one refers to the processes as \emph{workflows} \cite{Becker1999Identifying,Hollingsworth1995Wfmc}. The \ac{WfMC} defines workflows as ``the computerized facilitation or automation of a business process, in whole or part'' \cite{Hollingsworth1995Wfmc}.

    \emph{Process activities} are the atomic steps that processes consist of. The \ac{WfMC} differentiates between \emph{manual activities} and \emph{workflow activities}. The former are activities that involve user interaction in order to be completed, while the latter are automated and require no interaction \cite{Hollingsworth1995Wfmc}. As the term ``workflow activity'' might be misunderstood as ``any activity belonging to a workflow'', in the following the term \emph{automated activity} will be used instead.
  % subsection workflow (end)

  \subsection{Process definition} % (fold)
  \label{sub:process_definition}
    In order to be able to execute workflows, the underlying business processes must be machine processable and thus have to be formalized to an abstracted model \cite{Hollingsworth1995Wfmc}. This model is usually called \emph{process definition} and stored in form of some high-level programming language construct \cite{Hollingsworth1995Wfmc,Wutke2008Model}.
    The process definitions typically consist of a collection of activities with additional metadata such as associated applications or participants, and a set of rules which determine the execution order of these activities \cite{Hollingsworth1995Wfmc}. They further may contain references to other processes, which are treated as a single activity in the process definition \cite{Hollingsworth1995Wfmc,Casati1999Specification}.
  % subsection process_definition (end)

  \subsection{Process instance} % (fold)
  \label{sub:process_instance}
    A \emph{process instance} is an enactment of a process definition. A process definition may be instantiated multiple times, even at the same time. \cite{Casati1999Specification}. If only the automated parts of such an instance are meant, the \ac{WfMC} advocates the term \emph{workflow instance} \cite{Hollingsworth1995Wfmc}.

    Process instances have several states. When they are created, they are in the \emph{initiated} state. In this state, all relevant data has been provided, but the execution has not yet begun, \eg because not all requirements are met. When the process is started, it enters the \emph{running} state and it's activities may be started according to the process definition. If it has one or more instanciated activities, a process instance is in the \emph{active} state. Process instances may be suspended, \ie they enter the \emph{suspended} state and no activities are instanciated until they leave it again. There are two states that a stopped process instance can be in. Either the completion requirements are met and the stopped process instance is in the \emph{completed} state. Or the process instance stopped before its regular end, \ie because of an error or manual interruption. In this case the process instance is in the \emph{terminated} state \cite{Hollingsworth1995Wfmc}.

  % subsection process_instance (end)

  \subsection{Activity instance} % (fold)
  \label{sub:activity_instance}
    Like processes, activities are instanciated during workflow execution and have a set of states that they may be in. When an activity instance is created, it is in the \emph{inactive} state. From this state, it may enter the \emph{suspended} state, in which it will neither be activated nor assigned a worklist item. If the activity instance is not suspended, it is activated once its entry conditions are fulfilled. It then is in the \emph{active} state. When the execution of the activity has finished, it enters the \emph{completed} state \cite{Hollingsworth1995Wfmc}.
  % subsection activity_instance (end)

  \subsection{Workflow data} % (fold)
  \label{sub:workflow_data}
    In a \ac{WfMS}, several forms of data are distinguished, as they serve different purposes. The \ac{WfMC} differentiates between three types of data: workflow relevant data, workflow application data, and workflow control data \cite{Hollingsworth1995Wfmc}.

    \acp{WfMS} use \emph{workflow relevant data} to determine a process instance's status and the next activity to be executed. It is normally available to the \ac{WfMS} and both process- and activity instances \cite{Hollingsworth1995Wfmc}. \\
    Applications that are part of an workflow may work on domain specific data, which is called \emph{workflow application data}. In most cases, the \ac{WfMS} does not interact with this data other that providing it to the respective applications and limit access to it according to some authorization rules \cite{Hollingsworth1995Wfmc,Casati1999Specification}. \\
    Data that is internally managed by a \ac{WfMS} is refered to as \emph{workflow control data}. This data usually comprises the states of process- and activity instances and other internal statuses and is per se not interchanged in its default form \cite{Hollingsworth1995Wfmc,Casati1999Specification}.

    Russel et al. differentiate seven commonly used forms of data visibility in \acp{WfMS} \cite[p.~6-15]{Russell2005Workflow}:
    \begin{itemize}[nosep]
      \item \textbf{Activity Data} \hfill \\
        Data which is defined within an activity and which is accessible within the instance of this activity.
      \item \textbf{Sub-workflow Data} \hfill \\
        Data which is defined within a sub-workflow activity and is accessible from everywhere within this sub-workflow.
      \item \textbf{Scope Data} \hfill \\
        Data which is accessible within a subset of activities in a worfklow instance.
      \item \textbf{Multiple Instance Data} \hfill \\
        Data which is defined within an activity that can be instanciated multiple times. Each instance can access its own version of that data.
      \item \textbf{Workflow Instance Data} \hfill \\
        Data which is specific to a process instance of a workflow and which can be accessed by all components of that workflow during its execution.
      \item \textbf{Workflow Data} \hfill \\
        Data elements which are accessible to all components of all instances of a workflow and are controlled by the \ac{WfMS}.
      \item \textbf{Environment Data} \hfill \\
        Data which exists in the operating environment and which can be accessed by components of any workflow during execution.
    \end{itemize}

    Russel et al. identified six further types of data interaction between the various hierarchy levels in workflows \cite[p.~16-24]{Russell2005Workflow}:
    \begin{itemize}[nosep]
      \item \textbf{Activity -- Activity} \hfill \\
        Data is passed between two activity instances which belong to the same workflow instance.
      \item \textbf{Sub-workflow Activity -- Sub-workflow Components} \hfill \\
        Data is passed from a sub-workflow activity instance to the corresponding sub-workflow.
      \item \textbf{Sub-workflow Components -- Sub-workflow Activity} \hfill \\
        Data is passed back from a sub-workflow instance to the corresponding sub-workflow activity instance.
      \item \textbf{Activity -- Multiple Instance Activity} \hfill \\
        Data is passed from an activity instance to a successor activity which may be instanciated multiple times. It may be passed to all instances of the multiple instance activity or distributed among them according to specific rules.
      \item \textbf{Multiple Instance Activity -- Activity} \hfill \\
        Data is passed from an activity which may be instanciated multiple times to a successor activity instance.
      \item \textbf{Workflow Instance -- Workflow Instance} \hfill \\
        Data is passed from one instance of a workflow during its execution to another workflow instance that is being executed in parallel.
    \end{itemize}

    Workflow data may either be made available from a common datastore, get passed along with the control flow of a workflow, or be explicitly passed to the receiving component \cite[pp.~16-21]{Russell2005Workflow}.
  % subsection workflow_data (end)

  \subsection{Workflow participant and worklist} % (fold)
  \label{sub:workflow_participants}
    There are workflows that contain activities which require user interaction. A \ac{WfMS} thus provides the functionality to assign workflows and activities to workflow participants. The assignment can either be a specific one, targeting an individual person, or be more general, targeting a set of users from which the \ac{WfMS} may choose during execution time. These sets are usually based on an organizational structure that manifests itself in roles, of which a user may have one or more \cite{Hollingsworth1995Wfmc,Casati1999Specification}.

    Each user owns a so called \emph{worklist} that consist of activities to which he or she is assigned to and which are scheduled for execution. Depending on the actual implementation, activites may appear on multiple users' worklists until one of them signals that he or she will work on it \cite{Hollingsworth1995Wfmc,Casati1999Specification}.
  % subsection workflow_participants (end)

% section concepts (end)

\section{Typical architecture} % (fold)
\label{sec:typical_architecture}
  For large and complex organizations, the need arises to manage the creation, distribution and execution of workflows in a structured manner. An information system is a \ac{WfMS} if
  \begin{itemize}[nosep]
    \item it is able to define, create and manage the execution of workflows by using software that runs on one or more workflow engines;
    \item it is able to interpret process definitions;
    \item it can interact with involved participants; and
    \item it may invoke external applications \cite{Lawrence1997Workflow}.
  \end{itemize}

  According to the \ac{WfMC}, a workflow management system is ``a system that defines, manages and executes workflows through the execution of software whose order of execution is driven by a computer representation of the workflow logic'' \cite{Hollingsworth1995Wfmc}. The components of this system interlock in order to provide the overall functionality of a \ac{WfMS}. In the following, the typical characteristics of \ac{WfMS} architectures identified by the \ac{WfMC} are presented.

  \subsection{Functional areas} % (fold)
  \label{sub:functional_areas}
    The \ac{WfMC} divides the responsibilities of a \ac{WfMS} in three functional areas: \emph{build-time} functions, \emph{run-time process control} functions and \emph{run-time activity interaction} functions \cite{Hollingsworth1995Wfmc, Alonso1997Functionality}.

    The \emph{build-time} functionalities are concerned with the abstraction of workflows, \ie the creation of process definitions.
    The \emph{run-time process control} functionalities of a \ac{WfMS} are dealing with instanciating and controlling processes, coordinating the execution of activities within a process instance, initiating (but not performing) both participant interaction and application invocation \cite{Hollingsworth1995Wfmc}.

    Some activites require users to enter data or applications to perform a specific task. The \emph{run-time activity interaction} functions of a \ac{WfMS} provide the possibilities to do so. They make forms available to users, instruct other applications, and collect any resulting outcomes \cite{Hollingsworth1995Wfmc}.
  % subsection functional_areas (end)

  \subsection{System components} % (fold)
  \label{sub:system_components}

    The \ac{WfMC} identified four high-level groups of software components that most \acp{WfMS} have in common: \emph{Process Definition Tools}, \emph{Administration and Monitoring Tools}, \emph{Workflow Client Applications}, and \emph{Workflow Enactment Service} \cite{Hollingsworth1995Wfmc}.

    \textbf{Process definition tools} are designed for analysis, modelling, description and documentation of business proceses. The output of process definition tools -- process definitions -- can be interpreted by workflow engines in order to enact the respective workflow. The \ac{WfMC} notes, that process definition tools do not necessarily have to be part of a \ac{WfMS}, since the definition may take place in another tool as long as it is passed along in a standardized format \cite{Hollingsworth1995Wfmc}.

    The \textbf{administration and monitoring tools} are responsible for high-level monitoring and control of the system. Their functionalities may include user management, role management, logging, performance auditing, resource control, and supervision over running processes.

    The core function of the \textbf{workflow client applications} is to let the user retrieve worklist items that were assigned to him/her. In the \ac{WfMC} reference model they are thus sometimes referred to as \emph{worklist handlers} \cite{Hollingsworth1995Wfmc}.
    Yet, the \ac{WfMC} stresses that their functionality may be much broader, \eg letting him/her enter data that is associated to one worklist item, allow him/her to alter the worklist, signing in or off, or control the processes' statuses. The \ac{WfMC} thus advocates for the term \emph{workflow client applications} \cite{Hollingsworth1995Wfmc}.
    The user interface may be part of the workflow client applications or exist as a separate software component.

    In order to enact workflows, instances of them are created based on the interpretation of previously created process definitions. Workflow instances are usually managed by a component which is called \textbf{workflow engine}. The workflow engine decides which activities and sub-workflows of a workflow can be started, determines suitable participants, invokes external applications and it updates the users' worklists accordingly. It further manages the storage and flow of workflow control data and workflow relevant data \cite{Hollingsworth1995Wfmc}.

    The \textbf{workflow enactment service} groups one or more workflow engines into one logical component that exposes a single coherent external interface to other software \cite{Hollingsworth1995Wfmc}.
  % subsection system_components (end)
% section typical_architecture (end)
