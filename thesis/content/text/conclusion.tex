% -*- root: ../../main.tex -*- %

In the thesis at hand, mechanisms and implementation possibilities that Docker and its related tools can provide for \acp{WfMS} were proposed and discussed. In particular, the focus lay on figuring out how Docker can support the deployment and execution of workflows in a distributed environment, as well as on which decisions in software architecture and software design of a WFMS are complemented by Docker's functionality.

After introducing the concepts of \acp{WfMS}, Docker and its related tools, as well as prevalent architecture styles, variants for the utilization of Docker to enact workflows were explored. One such variant that features the encapsulation of all workflow elements in separate images as well as a shared data volume for the resulting containers was selected for implementation.

Thereafter, the attention was directed to the question as to how Docker can be used to create a flexible, failure resilient \ac{WfMS}. Given the choice between monolithic architecture, \ac{SOA}, and \ac{MSA}, it was found that the benefits which Docker has to offer can be reaped best in combination with a \ac{MSA} and inter-service communication using \ac{MOM}. Consequently, service boundaries for the micro-services were derived from the \ac{WfMS} reference model by the \ac{WfMC}.

The outcome of these considerations was manifested in the design and implementation of a prototype. In the course of both design and implementation of the prototype, artifacts were created that can be the starting point for both future research and \ac{WfMS} development. It was demonstrated that the prototype can be used to model a workflow and automatically distribute it to other machines in the form of images.

With regards to the variants that were deemed promising in \ref{sub:promising_combinations_of_characteristics}, follow-up research could be done on how these variants compare to each other as regards performance. Possible performance gains through the use of a compiled language instead of the used language Ruby could also be of interest. Further, the utilization of checkpoint and restore techniques to enable the migration of paused containers across nodes could be investigated \cite{Kim2015Checkpoint,Merker2015How}.

To conclude, it could be demonstrated that Docker and its ecosystem offer useful functionality for both \acp{WfMS} themselves as well as for the enactment of workflows – and is thus to be considered a fitting option for software architects that are charged with building \acp{WfMS} for enterprises that are confronted with the various challenges of the digital revolution.
