% -*- root: ../../main.tex -*- %

In the thesis at hand, mechanisms and implementation issues that arise from the utilization of Docker and its related tools in the context of \acp{WfMS} were proposed and discussed. Two general application areas for Docker were identified: Docker may influence the architecture and design of \acp{WfMS} as well as provide new ways to distribute and execute workflows.

After introducing the concepts of \ac{WfMS}, Docker and prevalent architecture styles, possibilities for the utilization of Docker for the enactment of workflows were explored. One of them which features the encapsulation of all workflow elements in separate images and a shared data volume, was chosen to be implemented. The attention was then centered on how Docker can be used to create a flexible, failure resilient \ac{WfMS}.

The outcome of these considerations was manifested in the design and implementation of a prototype in which the feasibility of a selection of these mechanisms was demonstrated. In the course of both design and implementation of the prototype, artifacts were created that can be used by succeeding researchers or \ac{WfMS} developers.

results:
  - three promising combinations for execution
  - capability table
  - WfMC model translated to docker-enabled microservices
  - statically compiled activity/workflow would be faster

outlook:
  - pause + move containers: http://blog.circleci.com/checkpoint-and-restore-docker-container-with-criu/
  - evaluate supported patterns?  http://www.workflowpatterns.com/documentation/documents/BPM-06-22.pdf
  - implement resource management

