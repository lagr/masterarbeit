% -*- root: ../../main.tex -*- %

In the thesis at hand, mechanisms and implementation issues that arise from the utilization of Docker for deployment and execution of \acp{WfMS} were proposed and discussed. Two general application areas for Docker were identified: on the one hand Docker may influence the architecture and design of \acp{WfMS}, on the other hand, it can provide new ways to distribute and execute workflows.

The

results:
  - three promising combinations for execution
  - capability table
  - WfMC model translated to docker-enabled microservices
  - statically compiled activity/workflow would be faster

outlook:
  - pause + move containers: http://blog.circleci.com/checkpoint-and-restore-docker-container-with-criu/
  - evaluate supported patterns?  http://www.workflowpatterns.com/documentation/documents/BPM-06-22.pdf
  - implement resource management
