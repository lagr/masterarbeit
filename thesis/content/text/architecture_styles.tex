% -*- root: ../../main.tex -*- %
Developers of software systems have to cope with challenging factors such as high complexity within their systems, an increased need for integration of internal and external functionality and evolving technologies. Several architectural approaches emerged from the attempt to cope with these challenges. Strîmbei et al. consider \emph{monolithic architecture}, \emph{\ac{SOA}} and \emph{\ac{MSA}} to be the most relevant ones \cite[p.~13]{Strimbei2015Software}. In \ref{sub:choice_of_an_achitecture_model}, an architecture for the prototype is chosen based on the properties of the different architectures that are presented in this chapter.

\paragraph{Monolithic Architecture} % (fold)
\label{par:monolithic_architecture}
Monolithic software systems are characterized by their cohesive structure. Usually, components in a monolith are organized within one programm, often running in one process \cite[p.~35]{Stubbs2015Distributed}. They communicate through shared memory and direct function calls. Monolithic applications are typically written using one programming language \cite[p.~14]{Strimbei2015Software}. In order to cope with increasing workload on a monolithic system, multiple instances of it are run behind a load balancer \cite[p.~35]{Stubbs2015Distributed}.

The strengths of monolithic architecture lie mostly in its comparably simple demands towards the infrastructure. As the application is run as one entity, deployment and networking are rather simple \cite[p.~35]{Stubbs2015Distributed}. Since data can be shared via memory or disk, monolithic applications can access it faster than it would be the case with networked components \cite[p.~14]{Strimbei2015Software}. \\
Also, as the interaction between the application's components happens XYZ, the complexity of this interaction is lower compared to interaction between distributed components \cite[p.~14]{Strimbei2015Software}.

The weaknesses of monolithic architecture originate from its cohesive nature. As its components are usually tightly coupled, changes to one component can affect other parts of the application, which complicates the introduction of new components as well as the refactoring of existing ones \cite{Stubbs2015Distributed}.
Components cannot be deployed individually, which hinders reuse of functionality across several applications more difficult and makes scaling of single bootleneck components impossible \cite{Stubbs2015Distributed}. Also, if the application runs in a single process, the failure of one component may bring down the whole application \cite[p.~5]{Newman2015Building}.
% paragraph monolithic_architecture (end)

\paragraph{Service-oriented Architecture} % (fold)
\label{par:service_oriented_architecture}
\ac{SOA} is based on the idea that code which provides related business functions can be bundled into one component that offers said functionality to other systems \emph{as a service}, thus avoiding duplicated implementation of the functionalities among these systems \cite[p.8]{Hohpe2004Enterprise}.
An application may then use several services in order to fulfill its own business function \cite[p.~390]{Papazoglou2007Service}.
The \ac{OASIS} describes \ac{SOA} as an architectural paradigm that supports the organization and usage of these services \cite{Standards2006Reference}. Each service provider exposes its offered services in a standardized way, \eg using \ac{WSDL}, which can then be utilized by \emph{service consumers} \cite[p.~390]{Papazoglou2007Service}, \cite[p.~17]{Strimbei2015Software}.

Messages between services in \ac{SOA} are either of direct nature, which is called point-to-point connection, or backed by a message bus, the \ac{ESB} which incorporates the integration logic, \eg on transport and transformation of messages, between services and supports asynchronous messages \cite[p.~393]{Papazoglou2007Service}. While the former leads to tight coupling between the components, which becomes impractical with increasing numbers of endpoints, the latter manages this scenario better \cite[p.~393]{Papazoglou2007Service}.

On the one hand, \ac{SOA} has some advantages in comparison to monolithic architecture.
Service consumers do not have to make assumptions - or know - how services work, they only have to rely on the invocation of a service and its result to be formed as expected \cite[p.~390]{Papazoglou2007Service}. As long as the interface and the output of existing services do not change, a service provider may thus be altered or its capabilities be extended without affecting its services' consumers \cite[p.~390]{Papazoglou2007Service}. \ac{SOA} thus enhances an organization’s ability to respond quickly to changes \cite[p.~390]{Papazoglou2007Service}, \cite[p.~254]{Choi2010Implementing}. \\
Since legacy applications can be provided with appropriate interfaces, SOA can help to integrate and extend them \cite[p.~390]{Papazoglou2007Service}.

On the other hand, \ac{SOA} has some drawbacks, too.
For example, the failure of a single service provider may bring down multiple applications that consume its services, if no fallback measures are in place \cite[p.~408f]{Papazoglou2007Service}.
Also, the overall performance of an application with \ac{SOA} depends on the aggregated performances of the services it uses and their respective interactions \cite[p.~408f]{Papazoglou2007Service}.

% paragraph service_oriented_architecture (end)

\paragraph{Micro-services Architecture} % (fold)
\label{par:micro_services_architecture}
The concept of \ac{MSA} is closely related to that of \ac{SOA}, as it also promotes the encapsulation of functionality in standalone services which can be used by other parts of a system. There is ambiguity whether \ac{MSA} is actually a concept on its own -- or rather a specialized application of \ac{SOA} \cite[p.~35]{Stubbs2015Distributed}, \cite[p.~17]{Strimbei2015Software}.
Stubbs et al. describe \ac{MSA} as a distributed system that consists of independent services which are   narrowly focused and thus considered ``lightweight'' \cite[p.~35]{Stubbs2015Distributed}.
That exact principle has been described as a version of \ac{SOA} before \cite[p.~395]{Papazoglou2007Service}.

Strîmbei et al. created a differentiation between SOA and MSA as a distinct concept based on several sources. They come to the conclusion, that while the communication in \ac{SOA} is synchronous and ``smart but dependency-laden'', \ac{MSA} usually relies on asynchronous, ``dumb, fast messaging'' -- meaning that there is few information on the participating services contained in the messaging infrastructure. Further, they perceive applications in \ac{SOA} to be typically imperative in their programming style, while \ac{MSA} would be in an event-driven programming style \cite[pp.~17-20]{Strimbei2015Software}. They see \ac{SOA} applications as being usually stateful and \ac{MSA} applications as stateless. Finally they characterize the databases in \ac{SOA} as large relational databases and the databases in \ac{MSA} as small, often non-relational databases.

One benefit of \ac{MSA}, which it shares with \ac{SOA}, is that each service can be developed in a language and with a toolset that suits its specific needs, \eg a lower-level language for time-critical but simple tasks or a high-level language with some framework for complex ones, instead of having to find a compromise that suits most of the application  \cite[p.~35]{Stubbs2015Distributed}, \cite[p.~4]{Newman2015Building}, \cite[p.~113]{Thones2015Microservices}.
The narrow focus of each service makes it less specialized to certain uses, which should theoretically enable better reuse of code \cite[p.~35]{Stubbs2015Distributed}.
Another positive aspect is the \emph{resilience} of micro-services when it comes to service failures, that is, a single failing service does not render the whole system incapable of working \cite[p.~5]{Newman2015Building}.
Due to the properties of the \ac{MSA}, micro-services may be deployed, upgraded and scaled individually \cite[p.~116]{Thones2015Microservices}.

There are also disadvantages that can emerge with the use of \ac{MSA}.
While \ac{MSA} facilitates deploying parts of an application individually, the overall amount of work required for the deployment of all services is higher and the coordination of the deployments is complexer than the deployment of a monolithic application \cite[p.~35]{Stubbs2015Distributed}.
As services may be unavailable at times, a mechanism has to be in place that allows the discovery of services \cite[p.~35]{Stubbs2015Distributed}.
Unless a dedicated logging service is introduced and used, there is no central access to the services' logs. Aggregation and analysis of errors and fixing them is thus more complicated in comparison to monolithic architecture \cite[p.~35]{Stubbs2015Distributed}.
In order to define the different services, it is necessary to find the right size for each service, \ie the appropriate scope of its functionality. This process poses a challenge that is not needed to this extent in monolithic architectures or \ac{SOA}.
% paragraph micro_services_architecture (end)
